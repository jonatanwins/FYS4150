
If we denote the vector of unknowns in Eq. \ref{discretization} as \( \mathbf{v} = [v_1, v_2, \ldots, v_n]^T \), the system of equations can be represented in matrix form as:


\begin{align}
    A \mathbf{v} = \mathbf{g}
    \label{Avg}
\end{align}

where
$$
    \g = \begin{pmatrix}
(\Delta x)^2 f_1 + v_0 \\
(\Delta x)^2 f_2 \\
(\Delta x)^2 f_3 \\
\vdots \\
(\Delta x)^2 f_{n-1} \\
(\Delta x)^2 f_n + v_{n+1}
\end{pmatrix}

$$

%\( \mathbf{g} = \frac{100}{n^2} [e^{\frac{-10}{n}}, e^{\frac{-20}{n}}, \ldots, e^{-10}]^T \) 


and \( \mathbf{A} \) is an \(n \times n\) tridiagonal matrix of the form:

\[
\mathbf{A} = \begin{pmatrix}
2 & -1 & 0 & \cdots & 0 \\
-1 & 2 & -1 & \cdots & 0 \\
0 & -1 & 2 & \cdots & 0 \\
\vdots & \vdots & \vdots & \ddots & \vdots \\
0 & 0 & 0 & \cdots & 2
\end{pmatrix}
\]

This matrix \( \mathbf{A} \) has the subdiagonal, main diagonal, and superdiagonal specified by the signature \( (-1, 2, -1) \). Specifically, the elements of \( \mathbf{A} \) are related to the coefficients in the discretized Eq. \ref{discretization} as follows:

- The main diagonal elements \( A_{ii} \) correspond to the coefficient of \( b_i \), which is \( 2 \).

- The subdiagonal elements \( A_{i,i-1} \) and the superdiagonal elements \( A_{i,i+1} \) correspond to the coefficients of \( a_{i} \) and \( c_{i} \), respectively, which are \( -1 \).

Therefore, the matrix equation for the discretized system is:

\[
\begin{pmatrix}
2 & -1 & 0 & \cdots & 0 \\
-1 & 2 & -1 & \cdots & 0 \\
0 & -1 & 2 & \cdots & 0 \\
\vdots & \vdots & \vdots & \ddots & \vdots \\
0 & 0 & 0 & \cdots & 2
\end{pmatrix}
\begin{pmatrix}
v_1 \\
v_2 \\
v_3 \\
\vdots \\
v_n
\end{pmatrix}
=
\begin{pmatrix}
g_1 \\
g_2 \\
g_3 \\
\vdots \\
g_n
\end{pmatrix}
\]

where the vector \( \mathbf{g} \) on the right-hand side represents the source terms modified by the grid spacing \( h^2 f_i \).